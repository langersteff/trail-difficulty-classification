% This is samplepaper.tex, a sample chapter demonstrating the
% LLNCS macro package for Springer Computer Science proceedings;
% Version 2.20 of 2017/10/04
%
\documentclass[runningheads]{llncs}
%
\usepackage{graphicx}
% Used for displaying a sample figure. If possible, figure files should
% be included in EPS format.
%
% If you use the hyperref package, please uncomment the following line
% to display URLs in blue roman font according to Springer's eBook style:
% \renewcommand\UrlFont{\color{blue}\rmfamily}
	
\begin{document}
%
\title{Difficulty classification of mountainbike trails utilizing deep neural networks}
%
%\titlerunning{Abbreviated paper title}
% If the paper title is too long for the running head, you can set
% an abbreviated paper title here
%
\author{Stefan Langer\inst{1}}
%
\authorrunning{F. Author et al.}
% First names are abbreviated in the running head.
% If there are more than two authors, 'et al.' is used.
%
\institute{LMU Munich
\email{stefan.langer@ifu.lmu.de}\\
\url{http://www.mobile.ifi.lmu.de} }
%
\maketitle              % typeset the header of the contributionY>
%
\begin{abstract}
The difficulty of mountainbike downhill trails is a subjective perception. 
There are multiple classifications of trails such as a Singletrail-Scale (S1-S6) or a color scale (blue, red, black), which are mostly used in dedicated mountainbike parks. 
We propose a deep neural network pipeline to classify trails into the latter. 
We generate an open dataset of acceleration data from multiple sensors, that were connected to the mountainbike frame as well as the rider.
We train a 2D convolutional neural network with a stacked and concatenated representation of the aforementioned acceleration signals.
We achieve an accuracy of 90\% on our test data and a 100\% accuracy on riders subjective feeling.
As to our knowledge this is the first work utilizing deep neural networks for mountainbike sports analytics.


\keywords{Sports analytics  \and Deep neural networks \and Accelerometer \and Convolutional neural networks.}
\end{abstract}
%
%
%
\section{Introduction}
Mountainbiking is a sport originated in X in year Y
Many types of riding, from more uphill oriented to downhill oriented
This paper focuses on the latter
There are multiple approaches in difficulty classification
e.g. Singletrail skala, which is more european, from S1 to S6, examples can be seen here
Singletrail skala is more valuable for non-bikepark tracks
On the otherhand classification into colors. Blue easy, red intermediate, black hard, double black diamond very hard. 
Can't do the latter
However, ratings are generally very subjective
This paper is supposed to learn from subjective labels and apply those on other tracks.
With our approach hoping to find the consensus between multiple subjective labels.


\subsection{Sensor data analysis - state of the art}
To our knowledge there is no scientific work towards difficulty classification of mtb trails
However, there is a great amount of work towards sensor based analyses with classical machine learning.
Classical approaches need domain specific knowledge, and handcrafted feature engineering
In recent years there has been a shift towards deep learning
One well researched section of sensor data analysis is the field of activity recognition
A lot of work has gone into classical, handcrafted approaches
Many papers using convolutional networks though. 
Which is the approach we follow

\subsection{Sports analytics (in mountainbiking)}
Telemetry for how much braking, suspension and so on
Rider specific data (heartrate etc.)
This is more of an athelete centered research
This work is more venue centered

Sensor data in sports are widely used
e.g. starting with smart devices like smartwatches etc. that keep track of rides
Apps like Strava, which offers a comparison in speed
However, no comparison in difficulty
One could use this work to automatically tell the difficulty of a ride, which would make things more comparable. 
This could also be applied to other sports.

Some applications make use of the singletrail skala to show the difficulty of rides. 
One such app would be Komoot. 
Utilizing Open Street Maps, the application knows which part of the tracks have which difficulty, since they are labeled within those maps.
This knowledge can then be aggregated to an overall difficulty of a ride.




\section{The OMTB dataset}

\subsection{Collecting and labeling data}
Sensor Units

Mounting on bike and helmet



We collected data in multiple types of locations, namely bikeparks and open trails (hiking paths)
For bikepark we can use the difficulty given by the operator. 
For gaps inbetween sections, we label the connection bits respectively.
Mostly 0 because the connectors are fire roads normally
For singletracks that are not in operated bikeparks, we use singletrail skala values given by Open Street Maps and map them to our colored label-scheme respectively.
Table X shows the mappings. We leave out difficulties above S3, since they were not rideable by our non-professional riders. 
Those singletracks have connection bits inbetween as well, which we label through OSM also
For bits, that the riders named to be obviously more or less difficult than the given rating we down- or upgraded the rating. (TODO: This feels wrong)

\subsection{Data representation}
We have 2 units of which we use 3 sensors each
First sensor accelerometer, which shows acceleration in 3 axes (x, y, z) in the unit g
This sensor sends in 12.500Hz
Second sensor is gyroscope, which shows rotational speeds in 3 axes (x, y, z) in the unit deg/s
This sensor sends in 25.000Hz
Third sensor is the barometer, which shows the relative difference in air pressure, hence metres above sea level
This sensor sends in 250Hz

Syncing and interpolating



\section{Classification through a 2D convolutional neural network}


\subsection{Input format}
What does the resulting "image" look like?

\subsection{Network architecture}
Why which window size?
Why which kernel size?
Net summary

% TODO: remove subsections if unnecessary

\section{Evaluation}


\subsection{Prediction on a labeled test data set}

\subsection{Subjective feeling of riders}

\subsection{Automated clustering of recordings}

\section{Discussion}

In this work, we proposed an end-to-end approach to classify mountainbike trails regarding their difficulty. 
We collected acceleration data on mountainbike rides, which we released under the name OMTB dataset for further exploration.
For the classification, we utilized a 2D convolutional neural network which achieved 90\% accuracy on a labeled test set as well as 100\% on riders' subjective feel \cite{ref_article1,ref_lncs1,ref_book1}.
We hope to promote sports analytics for downhill mountainbiking with this work.
For future work we suggest to enhance the OMTB dataset in terms of sample quantity as well as sensor data dimensions. 
Other sensors that might be of interest are the gyroscope to detect relative rotations, the barometer to detect changes in elevation, heart rate sensors to detect excitement and more.
Because difficulty classifications in trail centres are subjective, it would be feasible to not use labeled data in training and perform  unsupervised clustering.
This could lead to a novel way of difficulty classification. 
In order to keep the known classes (such as blue, red, black) one would have to vote for the majority of labeled classes in one cluster.
Mounting on helmet causes anomalies when not riding
The barometer does not show any significant influence



%
% ---- Bibliography ----
%
% BibTeX users should specify bibliography style 'splncs04'.
% References will then be sorted and formatted in the correct style.
%
% \bibliographystyle{splncs04}
% \bibliography{mybibliography}
%
% TODO: Create a dedicated bib file
%
\begin{thebibliography}{8}
\bibitem{ref_article1}
Author, F.: Article title. Journal \textbf{2}(5), 99--110 (2016)


\end{thebibliography}
\end{document}
